% term_paper.tex

% preamble
\documentclass[10pt,letterpaper,notitlepage,draft]{article}

%\usepackage[latin1]{inputenc}
%\usepackage[T1]{fontenc}
\usepackage{geometry}
\usepackage{amsmath}
\usepackage{amssymb}
\usepackage{latexsym}
\usepackage{eufrak}
\usepackage{eucal}
\usepackage{amsthm}
\usepackage{array}
\usepackage[final]{listings}
%\usepackage{tikz}

% define theorems
\theoremstyle{definition}
\newtheorem{my_thm}{Theorem}
\newtheorem{my_clm}{Claim}
\newtheorem{my_ex}{Example}
\newtheorem*{my_def}{Definition}
\newtheorem*{my_note}{Note}

% make q.e.d. symbol black square
\renewcommand\qedsymbol{
    \ensuremath{\blacksquare}
}

% listings settings
\lstset{
    basicstyle=\ttfamily,
    numbers=left,
    numberstyle=\small,
%   showspaces=true,
%   showtabs=true,
    mathescape=true,
    frame=single
}

% my title is a command for title and author
% #1 title
% #2 author
%\newcommand\mytitle[2]{
%    \parbox{0.9\textwidth}{
%        \centering\bfseries\large #1\\
%        By #2\vspace{\baselineskip}
%    }
%}

\setlength\extrarowheight{2pt}

\vfuzz2pt % don't report over-full v-boxes if over-edge is small
\hfuzz2pt % don't report over-full h-boxes if over-edge is small

\pagestyle{empty}

\title{CS 527 Fall 2014\\Term Paper}
\author{Spencer Hubbard} % TODO: add other author's names on new line
\date{} % TODO: change date to actual due date

\begin{document}

% front matter
\maketitle

% TODO: write abstract
%\begin{abstract}
%\end{abstract}

% main matter
\section{Introduction}
For each $i \in \lbrace 1, \ldots, n\rbrace$, let $x_i$ be a symbol and let $p_i$ be the probability associated with $x_i$. 
Let $q_i = \sum_{j=1}^{i-1} p_j + p_i / 2$ and let $\ell_i = \lceil\log 1 / p_i \rceil + 1$. 
Now, let $c_i$ be the first $\ell_i$ bits in the binary expansion of the number $q_i$. 
Shannon-Fano-Elias code uses $c_i$ as the codeword for $x_i$. 
It can be shown that the length of the codewords satisfy Kraft's inequality, i.e., $\sum_{i=1}^n 2^{-\ell_i} \le 1$, which means that the code is uniquely decodable. 
It can also be shown that the code is a prefix code. 
Note that we do not assume that the symbols are indexed by decreasing probability. 
When the symbols are indexed by decreasing probability, this is just Shannon-Fano code.

Suppose that Alice wants to send a secret message to Bob but their adversary Eve---the eavesdropper---is allowed to intercept and attempt to read any message that Alice sends Bob. 
Suppose also that Alice, Bob, and Eve all know the set of symbols and their associated probabilities. 
Notice that for a fixed set of $n$ symbols and associated probabilities, there are $n!$ Shannon-Fano-Elias codes corresponding to the $n!$ permutations of the indices $\lbrace 1, \dots, n\rbrace$. 
This means if Alice and Bob can agree upon a particular permutation without Eve knowing, then Alice can use Shannon-Fano-Elias code to send a message to Bob and Eve must potentially check $n!$ permutations before successfully reading the message.

\section{Huffman Code vs Shannon Code}
% The downsides of huffman codes and why it can't be used for encryption

\section{Improved Shannon-Fano-Elias Coding}

\subsection{"Better" PMF for encoding}
%This improvement to Shannon-Fano-Elias code aims to change the PMF used for the encoding. Since Huffman coding is more optimial, we will use the length of each symbol to generate a new PMF.
Since the symbols are not necessarily indexed by decreasing probability, it follows that the resulting code may have a large average length.
To decrease the average length of the code, we will replace the true probabilities with some temporary probabilities and construct the code using these temporary probabilities.

For each $i \in \lbrace 1, \ldots, n\rbrace$, let $x_i$ and $p_i$ be defined as before.
Let $\ell_i$ be the length of the Huffman codeword for the symbol $x_i$ with probability $p_i$ and let $L$ be the average length of this Huffman code, i.e., let $L = \sum_{i=1}^n p_i \ell_i$.
Now, let $p_i^\prime = 2^{-\ell_i}$ be the temporary probability of the symbol $x_i$ and let $\ell_i^\prime$ be the length of the Shannon-Fano-Elias codeword for the symbol $x_i$ with the temporary probability $p_i^\prime$, i.e., let $\ell_i^\prime =  \lceil\log_2 1 / p_i^\prime \rceil + 1$.
Also, let $L^\prime$ be the average length of this Shannon-Fano-Elias code with respect to the true probabilities, i.e., let $L^\prime = \sum_{i=1}^n p_i \ell_i^\prime$.
Notice that $\ell_i^\prime = \ell_i + 1$.
This means
\begin{align*}
L^\prime
&= \sum_{i=1}^n p_i \ell_i^\prime \\
%&= \sum_{i=1}^n p_i \left(\lceil\log 1 / p_i^\prime \rceil + 1\right) \\
&= \sum_{i=1}^n p_i \left(\ell_i + 1\right) \\
&= \sum_{i=1}^n p_i \ell_i + \sum_{i=1}^n p_i \\
&= L + 1
\end{align*}
which means the average length of this Shannon-Fano-Elias code is within one of optimal.

% Another reason to use the better PMF 

\subsection{Codeword Truncation}

\begin{lstlisting}[breaklines]
Input: Shannon-Fano-Elias codeword set codeword(xi) for 1<=i<=n.
Output: codeword(xi) with lower expected length.
    for 1<=i<=n do
        c(xi)<- codeword(xi)
        while codeword(xi-1),c(xi) and codeword(xi+1) are mutually prefix-free do
            codeword(xi) <- c(xi)
            c(xi) <- removing the rightmost bit of the codeword(xi)
        end while
    end for
\end{lstlisting}

In this algorithm, each codeword has the rightmost bits are truncated while it and its preceding and succeeding codewords are mutually prefix-free. Since the algorithm removes bits from the codeword, it will reduce the average code length and therefore optimize the code. This works because the closer codes are, the more similar they are due to the codeword being based on the running probability. Therefore, the codeword does not have to be compared to every other codeword and only the ones immediately before and after it. 

Below we have a code with 4 codewords:

\begin{table}[h]
\begin{center}
\begin{tabular}{|c|c|c|l|}
\hline
Values & P(X) & F(X) & Codeword \\ 
\hline 
\hline
D & 1/4 & 1/8 & 001 \\
\hline
C & 1/6 & 1/3 & 0101 \\
\hline
B & 1/4 & 13/24 & 100 \\
\hline
A & 1/3 & 5/6 & 110 \\
\hline
\end{tabular}
\end{center}
\caption{caption text here}\label{t1}
\end{table}

Using the algorithm in order we have
001 compared to 0101.  
These numbers are mutually prefix-free and the rightmost value and be deleted.  
D is now 00.  
Removing one more would result in 0, which is not prefix free and the algorithm continues to the next value.

0101 is compared to 00 and 100.
These numbers are prefix free and the rightmost value can be deleted.
C is now 010.
010 is compared to 00 and 100.
These numbers are prefix free and the rightmost value can be deleted.
C is now 01.
Deleting one more would result and 0 and therefore a prefix, thus the algorithm can move to the next value.

100 is compared to 01 and 110.
These numbers are prefix free and the rightmost value can be deleted.
B is now 10.
Deleting one more value would result in 1, which is a prefix of 110 and therefore the algorithm moves on.

110 is compared to 10.
These numbers are prefix free and the rightmost value can be deleted.
A is now 11.
Deleting one more value would result in 1, which is a prefix of 10, therefore the algorithm continues.
The algorithm ends because our alphabet has been exhausted.

Our resulting code is 
\begin{tabular}{|c|c|c|c|}
\hline
Values & P(X) & F(X) & Codeword \\
\hline
\hline
D & 1/4 & 1/8 & 00 \\
\hline
C & 1/6 & 1/3 & 01 \\
\hline
B & 1/4 & 13/24 & 10 \\
\hline
A & 1/3 & 5/6 & 11 \\
\hline
\end{tabular}


Here we see a significantly reduced code.  The average length has been reduced to 2, as opposed to the 3.2 that it was previously.  Not only is this code significantly reduced in this case, it is also optimal.  Thus, in certain cases, this truncation method can result in an optimal Huffman code.


% TODO: write section
\section{Finding the Codeword Set Used for Encoding}

% TODO: write section
\section{Conclusion}

% back matter
%\nocite{*}
%\bibliographystyle{amsplain}
%\bibliography{reference}

\end{document}
